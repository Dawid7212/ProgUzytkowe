\documentclass[14pt, letterpaper, titlepage]{article}
\usepackage[left=3.5cm, right=2.5cm, top=2.5cm, bottom=2.5cm]{geometry}
\usepackage[MeX]{polski}
\usepackage[utf8]{inputenc}
\usepackage{graphicx}
\usepackage{enumerate}
\usepackage{amsmath} %pakiet matematyczny
\usepackage{amssymb} %pakiet dodatkowych symboli
\title{Pierwszy dokument LaTeX}
\author{Dawid Moeck}
\date{Październik 2022}
\begin{document}
\maketitle
Tu umieszczamy kod TeXa, który będzie kompilowany
\section{Pare słow o mnie}
Mam 20 lat, ukończyłem techikum informatyczne, urodziłem się na wsi stąd też znam się na rolnictwie lepiej niż na informatyce :D
\subsection{Dlaczego wybrałem sudia niestacjonarne?}
Wyrałem je ponieważ nie mogłem wybrać stacjeonarnych z powodu pracy w dni robocze.

\paragraph{paragraf pierszy} 
%Tutaj bedzie komentarz
Test paragrafu pierwszego \\
 Nowy wiersz
\subparagraph{Dlaczego wybrałem kierunek inforamtyka?}
\begin{enumerate}[A]
\item Interesuję się inforamtyką
\item Ten zwód to gawarancja dobrej przyszłości

\end{enumerate}


\appendix
\textbf{ pogrubienie tekstu} \newline
\textit{pochylenie tekstu} \\
\underline{podkreślenie tekstu} \\ 
%\fontsize{20pt}{2x}\selectfont tekst do powiększenia\\
%Zmiana wielkosci tekstu w trakcie \par


\abstract
Test strony streszczającej
\newpage
\title\textbf{Integracyjny piknik sportowy Razem w Aktywność!}
\paragraph{KS Ożarowianka Ożarów Mazowiecki zorganizował 25 września 2022 r. Integracyjny piknik sportowy Razem w Aktywność. Piknik był imprezą wspieraną przez Urząd Marszałkowski Województwa Mazowieckiego i Powiat Warszawski Zachodni, organizowaną w ramach Europejskiego Tygodnia Sportu. Uczestnikami pikniku były polskie i ukraińskie rodziny z Ożarowa Mazowieckiego i okolic.}
ldksagj
\subparagraph{Cele pikniku\newline}

Celem pikniku była integracja rodzin polskich i ukraińskich w sporcie, zachęcenie do aktywnego spędzania czasu wolnego i dobra zabawa dla dzieci. Pogoda dopisała, na pikniku przeprowadziliśmy wiele konkurencji sportowych, w których uczestniczyły setki dzieci. Pokazaliśmy jak wygląda życie zawodników trenujących w naszym klubie od podszewki i, jak zaktywizować rodziców, którzy u nas trochę "z automatu" zostali wciągnięci w sportową rywalizację i rozpoczęli zmagania w ramach Ligi Rodziców. W Lidze rywalizują rodzice dzieci trenujących w poszczególnych rocznikach akademii. Już kolejny sezon liga cieszy się ogromną popularnością wśród graczy i ma swoje grono kibiców.

\subparagraph{Atrakcje pikniku}
\textit{Na pikniku przeprowadzono wiele konkurencji sportowych:} 

\begin{enumerate}
\item Na pikniku przeprowadzono wiele konkurencji sportowych: odbył się I Ożarowski Trójbój Sportowy w ramach którego dzieci rywalizowały w rzucie piłką lekarską, biegu na 30 m i skoku w dal. Poza tym były wyścigi w workach, rzut woreczkiem do celu, rzuty karne z zawiązanymi oczami, które wywołały wiele uśmiechu.
\item Poza konkurencjami sportowymi odbyły się sparingi i treningi otwarte różnych roczników. Rywalizowaliśmy m.in. z drużynami RKS Ursus i drużyną Naprzód Stare Babice.
\item Atrakcjom sportowym towarzyszyły dmuchańce, malowanie buziek, wata cukrowa i popcorn.
\item Warsztaty kulinarne z zasad zdrowego odżywiania się sportowców
\end{enumerate}

\noindent
\textbf{Aspekty sportowe} \\

Podczas pikniku zaprezentowaliśmy potencjał naszych podopiecznych. Zainteresowane uprawianiem sportu dzieci mogły wziąć udział w treningach otwartych i obejrzeć sparingi wybranych roczników. Treningi i sparingi odbyły drużyny roczników 2007 Trenera Dariusza Kosińskiego, 2009A Trenera Piotra Biechońskiego, 2010B i 2013A Trenera Romana Karakoszki, 2013B Trenera Mateusza Krajewskiego, roczniki 2016A i 2017A trenera Dominika Kwaśniewskiego oraz 2016B Trenera Roberta Ludwikowskiego. Ponadto otwarte treningi motoryczne zawodników Akademii KS Ożarowianka poprowadził Trener Eryk Murawski.\\

– W Akademii KS Ożarowianka Ożarów Mazowiecki duży nacisk położony jest na rozwój motoryczny zawodników. Roczniki 2013 i starsze poza trzema jednostkami treningu piłkarskiego w tygodniu obowiązkowo uczestniczą w zajęciach motorycznych. Te treningi są bardzo ważne w rozwoju zawodników piłki nożnej – mówi Trener przygotowania motorycznego Eryk Murawski - Efekty poprawy wydolności, siły, zwrotności bardzo szybko widać na boisku. To powoduje, że zawodnicy regularnie uczęszczający na zajęcia, robią to chętnie.\\

Na zakończenie mecz rozegrali rodzice dzieci Akademii KSO i rodzice dzieci Akademii Naprzód Stare Babice. Było to emocjonujące spotkanie! \\

Liczymy na to, że piknik zachęci dzieci, młodzież i rodziców do aktywnego uprawiania sportu. Jako Klub jesteśmy w stanie spełnić szczególne wymagania uczestników naszych zajęć: moglibyśmy prowadzić zajęcia dla grupy dzieci z Ukrainy w języku rosyjskim, mamy możliwość prowadzenia zajęć w języku migowym, wśród naszej kadry są trenerzy, którzy ukończyli szkolenia w zakresie prowadzenia zajęć terapeutycznych dla dzieci zagrożonych depresją.\\
\end{document}